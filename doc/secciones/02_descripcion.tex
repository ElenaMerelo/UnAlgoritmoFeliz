\chapter{Descripción Del Problema}

En 2019, para la asignatura de metaheurísticas de la Universidad de Granada \cite{merelo_molina_2021} se propuso inventar una metaheuristica. En
ese momento me acababa de leer el libro "Un Mundo feliz" de Aldous Huxley, así que decidí adaptar el proceso de creación 
de humanos que describe el libro a un algoritmo evolutivo. A lo largo del desarrollo se me presentaron varios problemas. El
\textit{primer problema} fue el lenguaje escogido, lo desarrollé en Python, un lenguaje caracterizado por su fácil sintaxis pero su deficiencia de
velocidad. Cada ejecución del problema llevaba horas, por lo que se hacía muy difícil de ejecutar y analizar. El
\textit{segundo problema} fue la optimización de los parámetros, el algoritmo quedaba estancado en óptimos locales. Estos problemas son 
los que se quieren abordar en el desarrollo de este trabajo.

Una de las formas de conseguir altas prestaciones es escoger el lenguaje adecuado. Por ello se ha escogido un lenguaje que
hace poco entró en el \"Petaflop Club\", que incluye aquellos lenguajes que superan 1 petaflop/segundo como rendimiento pico. Se trata
de Julia \cite{julia}, un lenguaje de programación multiparadigma de tipado dinámico. Orientado a la computación técnica y 
científica. La rapidez fue uno de los principales objetivos a la de la creación del lenguaje. 

\section{Naturaleza del algoritmo}

Estamos hablando de un algoritmo basado en la evolución de una población, así que seguirá la estructura de los
algoritmos evolutivos. Siguiendo la definición dada por Goldberg \cite{goldberg89}, "los Algoritmos Genéticos son algoritmos de búsqueda
basados en la mecánica de selección natural. Combinan la supervivencia del más apto entre estructuras de secuencias con un intercambio de 
información estructurado, aunque aleatorizado, para construir así un algoritmo
de búsqueda que tenga algo de las genialidades de las búsquedas humanas".

Para alcanzar la solución a un problema se parte de un conjunto inicial de individuos, llamado \textit{población},
generado de manera aleatoria. Cada uno de estos individuos representa una posible solución al problema. Estos individuos
evolucionarán siguiendo el proceso propuesto en el libro de Aldous Huxley, y se adaptarán en mayor o menor medida,
tras el paso de cada generación, a la solución requerida.
