\chapter{Descripción Del problema}

Actualmente las ramas de la Ciencia de la computación y la Ingeniería del Software se encuentran muy separadas. A la
hora de desarrollar un algoritmo no se priorizan conceptos como arquitectura limpia \cite{cleanArquitecture2017} o si 
sigue estándares de código \cite{cleanCode2008}. Con este trabajo se quiere desarrollar un algoritmo que rompa con esta idea. 
Un algoritmo que no dependa de los datos de entrada.

El objetivo sería, por tanto, desarrollar una Metaheurística basada en el libro de Un mundo feliz de Aldous
Huxley pero que además cumpla con el modelo de "Software as a Service" o Software como Servicio \cite{SaS}.

\section{Naturaleza del algoritmo}

Estamos hablando de un algoritmo basado en la evolución de una población, así que seguirá la estructura de los
algoritmos genéticos. Siguiendo la definición dada por Goldberg \cite{goldberg89}, "los Algoritmos Genéticos son algoritmos de búsqueda
basados en la mecánica de selección natural. Combinan la supervivencia del más apto entre estructuras de secuencias con un intercambio de 
información estructurado, aunque aleatorizado, para construir así un algoritmo
de búsqueda que tenga algo de las genialidades de las búsquedas humanas".

Para alcanzar la solución a un problema se parte de un conjunto inicial de individuos, llamado \textit{población},
generado de manera aleatoria. Cada uno de estos individuos representa una posible solución al problema. Estos individuos
evolucionarán siguiendo el proceso propuesto en el libro de Aldous Huxley, y se adaptarán en mayor o menor medida,
tras el paso de cada generación, a la solución requerida.



