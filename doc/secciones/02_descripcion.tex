\chapter{Descripción del problema}

Actualmente las ramas de la Ciencia de la computación y la Ingeniería del Software se encuentran muy separadas, a la
hora de desarrollar un algoritmo nunca nos ponemos a pensar en cosas como arquitectura limpia o si sigue estándares
de código. Con este trabajo se quiere desarrollar un algoritmo que rompa con esta idea. Un algoritmo que no dependa
de los datos de entrada, y al que le puedas pasar la información, por ejemplo, a través de llamadas a una API.

El objetivo sería, por tanto, desarrollar una Metaheurística basada en el libro de "Un mundo feliz" de Aldous
Huxley pero que además cumpla con el modelo de "Software as a Service" o Software como Servicio.

\section{Naturaleza del algoritmo}

Estamos hablando de un algoritmo basado en la evolución de una población, así que seguirá la estructura de los
algoritmos genéticos. Siguiendo la definición dada por Goldberg, "los Algoritmos Genéticos son algoritmos de búsqueda
basados en la mecánica de selección natural y de la selección natural. Combinan la supervivencia del más apto entre estructuras
de secuencias con un intercambio de información estructurado, aunque aleatorizado, para construir así un algoritmo
de búsqueda que tenga algo de las genialidades de las búsquedas humanas"\cite{goldberg89}.

Para alcanzar la solución a un problema se parte de un conjunto inicial de individuos, llamado \textit{población},
generado de manera aleatoria. Cada uno de estos individuos representa una posible solución al problema. Estos individuos
evolucionarán siguiendo el proceso propuesto en el libro de Aldous Huxley, y se adaptarán en mayor o menor medida,
tras el paso de cada generación, a la solución requerida.

\section{Estructura}

En el libro se describe cómo se alcanza esta raza humana perfecta mediante una "cadena de montaje", y eso es lo que se
tratará de reflejar en la metaheurística a desarrollar.

El proceso comienza en la \textbf{\textit{Sala de Fecundación}}, aquí se crea la provisión de óvulos y se fecundan. Una
vez \textit{fecundados} los óvulos pasan a las incubadoras. Aquí es donde se decide la \textit{casta} a la que
pertenecerá cada individuo:

\begin{itemize}
    \item \textbf{Alphas}: son los más inteligentes, a este grupo pertenece la élite. Tienen responsabilidades y son
    los que tienen la capacidad de tomar decisiones.
    \item \textbf{Betas}: son menos inteligentes que los anteriores y su función principal se reduce a tareas
    administrativas.
    \item \textbf{Gammas}: son los empleados subalternos, cuyas tareas requieren de habilidad.
    \item \textbf{Deltas}:a este grupo pertenecen los empleados de los anteriores.
    \item \textbf{Epsilones}: es la casta inferior, a ella pertenecen los empleados para trabajos arduos.
\end{itemize}

Con esta estructura en mente la meta heurística se divide en las siguientes fases:

\begin{itemize}
    \item \textbf{Sala de Fecundación}: los individuos son creados de manera totalmente aleatoria. Tantos individuos
    como indique un parámetro \textit{I}.
    \item \textbf{Sala de incubación}: en esta fase realizamos la división en castas. Se hará basándonos en el valor de
    la función objetivo del individuo. Además cada casta tendrá un porcentaje diferente de la población.
    \item \textbf{Evolución de las castas}: cada casta seguirá un proceso de evolución diferente que definiremos más
    adelante.
\end{itemize}


