\chapter{Introducción}

A principios de este año leí el conocido libro de Aldous Huxley: Un mundo feliz. La novela es una
distopía que describe el desarrollo en tecnología reproductiva, cultivos humanos e hipnopedia y el manejo de las
emociones por medio de drogas. La población se ordena en \textbf{castas}, asignadas desde el nacimiento, donde cada uno
sabe y acepta su lugar en la sociedad. La guerra y la pobreza han sido erradicadas, y todos son permanentemente
felices.

Cuando hablamos de algoritmos genéticos nuestro objetivo es alcanzar la solución óptima de un problema, y este libro describe el 
proceso mediante el cual han alcanzado la raza humana perfecta. Por tanto el objetivo final de este trabajo es desarrollar
un algoritmo genético basándonos en el proceso de fecundación del libro y comparar su comportamiento con otros algoritmos. Además
investiga cómo la división en castas afecta la diversidad en la población.

Este proyecto es software libre, y está liberado con la licencia \cite{gplv3}.