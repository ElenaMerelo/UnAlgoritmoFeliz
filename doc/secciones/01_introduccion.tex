\chapter{Introducción}

A principios de este año leí el conocido libro de Aldous Huxley: Un mundo feliz. La novela describe una
distopía que describe el desarrollo en tecnología reproductiva, cultivos humanos  e hipnopedia y el manejo de las
emociones por medio de drogas. La población se ordena en castas que son asignadas desde el nacimiento donde cada uno
sabe y acepta su lugar en la sociedad. La guerra y la pobreza han sido erradicadas, y todos son permanentemente
felices. Sin embargo, para llegar a este estado se han sacrificado conceptos como la familia o la diversidad
cultural.

En el libro se muestra como se ha encontrado la forma de reproducirnos, mediante una cadena de montaje,
hasta crear la raza humana perfecta. Esto tiene un símil fácil en el campo de la Ciencia de la Computación: llegar
a la \textit{solución óptima}.

Este proyecto es software libre, y está liberado con la licencia \cite{gplv3}.