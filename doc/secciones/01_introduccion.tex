\chapter{Introducción}

En este capítulo se introduce la motivación del proyecto y los objetivos que se quieren alcanzar durante el desarrollo. 

\section{Motivación}

A principios de este año leí el conocido libro de Aldous Huxley: Un mundo feliz. La novela es una
distopía que describe el desarrollo en tecnología reproductiva, cultivos humanos e hipnopedia y el manejo de las
emociones por medio de drogas. La población se ordena en \textbf{castas}, asignadas desde el nacimiento, donde cada uno
sabe y acepta su lugar en la sociedad. La guerra y la pobreza han sido erradicadas, y todos son permanentemente
felices. Se trata de un ``mundo óptimo``', cuya optimización está basada en la población y en el equilibrio que crea
la división en castas, no en un único individuo 

Cuando hablamos de algoritmos evolutivos nuestro objetivo es alcanzar la solución óptima de un problema, y este libro describe el 
proceso mediante el cual han alcanzado la raza humana perfecta. Por tanto se quiere desarrollar un algoritmo evolutivo basado en 
el proceso de fecundación del libro y comparar su comportamiento con otros algoritmos. Además investigar cómo la división en castas afecta 
la diversidad en la población.

\section{Objetivos del trabajo}

\begin{enumerate}
    \item Mejorar la diversidad en problemas de optimización mediante la división de la población en castas.
    \item Desarrollar una herramienta de altas prestaciones para la ejecución de algoritmos evolutivos.
    \item Aplicar el desarrollo ágil en la ciencia \cite{DBLP}.
\end{enumerate}


A continuación se establece el problema que se aborda en el trabajo. Luego se describe el estado del arte en el campo 
de las metaheurísticas y algoritmos bio-inspirados. Posteriormente cómo se ha implementado el algoritmo. Continuando con exploración
y análisis de resultados para elegir los parámetros que mejor se adapten al algoritmo. Finalizando con las conclusiones y siguientes pasos.   