\chapter{Introducción}

En este capítulo se introduce la motivación del proyecto y los objetivos que se quieren alcanzar durante el desarrollo. 

\section{Motivación}

A principios de este año leí el conocido libro de Aldous Huxley: Un mundo feliz. La novela es una
distopía que describe el desarrollo en tecnología reproductiva, cultivos humanos e hipnopedia y el manejo de las
emociones por medio de drogas. La población se ordena en \textbf{castas}, asignadas desde el nacimiento, donde cada uno
sabe y acepta su lugar en la sociedad. La guerra y la pobreza han sido erradicadas, y todos son permanentemente
felices. Se trata de un ``mundo óptimo``, cuya optimización está basada en la población y en el equilibrio que crea
la división en castas, no en un único individuo.

Cuando hablamos de algoritmos evolutivos nuestro objetivo es alcanzar la solución óptima de un problema, y este libro describe el 
proceso mediante el cual han alcanzado la raza humana perfecta. Por tanto se quiere desarrollar un algoritmo evolutivo basado en 
el proceso de fecundación del libro y comparar su comportamiento con otros algoritmos. Además, investigar cómo la división en castas afecta
la diversidad en la población.

\section{Objetivos del trabajo} \label{sect:goals}

En este apartado se definen los objetivos del trabajo que se va a desarrollar.
\begin{enumerate}
    \item Mejorar la diversidad en problemas de optimización mediante la división de la población en castas. En
    la primera versión \cite{merelo_molina_2021}, el algoritmo quedaba estancado en óptimos locales. Según
    algunos autores, el estancamiento temprano de un algoritmo se debe a la falta de diversidad, se quiere comprobar
    si la división en castas ayuda a evitar este problema.
    \item Desarrollar una herramienta de altas prestaciones para la ejecución de algoritmos evolutivos. En una 
    primera versión del trabajo \cite{merelo_molina_2021} se vio que la metaheurística era lenta.
    \item Aplicar el desarrollo ágil en la ciencia \cite{DBLP} que asegure que la versión final del trabajo tiene calidad y
    es mantenible a lo largo del tiempo.
\end{enumerate}

Desarrollar una herramienta con metodología ágil permitirá que el trabajo esté seguro ante cambios. Al tratarse de un
trabajo de investigación con tantas piezas, asegura que todas funcionan entre sí y que al añadir nueva funcionalidad
no se rompe nada es muy importante. Lo que dará mucha confianza a la hora de hacer los cambios necesarios para alcanzar
los otros dos objetivos.

A continuación se establece el problema que se aborda en el trabajo. Luego, se describe el estado del arte en el campo
de las metaheurísticas y algoritmos bio-inspirados. Posteriormente se describirá cómo se ha implementado el algoritmo,
continuando con exploración y análisis de resultados para elegir los parámetros que mejor se adapten al algoritmo; 
finalizando con las conclusiones y siguientes pasos.