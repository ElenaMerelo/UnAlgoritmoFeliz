\chapter{Implementación}

- en base a los requisitos que se va a usar de forma concreta
- Calcula la diversidad - Con qué he calculado la diversidad

¿qué he utilizado, estructuras de datos, mutables inmutables, precision del float, que bibliotecas he usado?

\section{Metodología del desarrollo}

Un requisito fundamental a la hora de implementar es usar una metodología que garantice la evolución del código. Por eso la implementación 
del software se ha dividido en tareas siguiendo la metodología ágil Kanban. Se divide el proyecto en tareas pequeñas que aporten valor al producto 
final. De esta forma puedes probar cada tarea a medida que la vas acabando. Al final del desarrollo estás segura de que todas las piezas funcionan
correctamente.

Otro requisito es desarrollar un algoritmo que sea mantenible. Por tanto, a la hora del desarrollo se priorizarán conceptos 
como arquitectura limpia \cite{cleanArquitecture2017} y la limpieza del código \cite{cleanCode2008}, así como el desarrollo basado en tests. De esta manera, se asegura 
que el algoritmo está asentado sobre una base firme. Asimismo, gracias al desarrollo basado en tests para cada cambio que se haga 
se sabrá si se ha roto lo que estaba escrito hasta el momento. 

- qué he usado para conseguir los hitos
\section{Estructura del algoritmo}

Una de las formas de conseguir altas prestaciones es escoger el lenguaje adecuado. Por ello se ha escogido un lenguaje que
hace poco entró en el \"Petaflop Club\", que incluye aquellos lenguajes que superan 1 petaflop/segundo como rendimiento pico. Se trata
de Julia \cite{julia}, un lenguaje de programación multiparadigma de tipado dinámico. Orientado a la computación técnica y 
científica. La rapidez fue uno de los principales objetivos a la hora de crear el lenguaje. 

Para ayudar a la reutilización es muy importante que el código sea limpio, mantenible y haya sido testeado, por eso se ha decidido seguir los estándares de arquitectura y código marcados 
por Robert C. Martin \cite{cleanArquitecture2017, cleanCode2008}. Usar estándares va alineado con el objetivo de aplicar el desarrollo ágil a la ciencia.
