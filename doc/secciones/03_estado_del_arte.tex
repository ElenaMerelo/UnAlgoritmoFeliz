\chapter{Estado del arte}

En los últimos años ha sido escrita mucha literatura acerca de algoritmos bio-inspirados. Estos algoritmos imitan procesos biológicos o estructuras presentes
en la naturaleza, y se aplican para resolver problemas de optimización. Por ejemplo, la \textbf{Optimización por Nube de Partículas} o PSO (\textit{Particle Swarm Optimization})
que imita el comportamiento de las partículas en la naturaleza, o el \textbf{Algoritmo colonia de abejas} basado en el comportamiento de los enjambres de abejas
para encontrar la miel. Esta gran cantidad de metaheurísticas basadas en metáforas ha sido criticada por algunos autores como K. Sörensen\cite{https://doi.org/10.1111/itor.12001},
que considera que esta línea de investigación amenazada con separar el ámbito de las metaheurísticas del rigor científico. Además estos algoritmos suelen usar terminología inspirada por el dominio,
lo que conduce a nuevos términos que describen conceptos que ya estaban establecidos \cite{mitigating_metaphors}. Lo que hace que sea difícil entender como funcionan estos algoritmos
y la relación que tienen con otras metaheurísticas. Autores como Swan \cite{metaheuristics} establecen para que la investigación en el campo de las metaheurísticas evite la fragmentación y 
la falta de reproductibilidad, se necesita una infraestructura científica y computacional fuerte, donde diferentes enfoques puedan ser desarrollados, analizados y comparados. El algoritmo de este 
evita caer en el uso de metáforas para explicar su comportamiento. Se inspira en ellas para crear una estructura de datos que ayude a mantener la diversidad a lo largo de toda la ejecución.

Otro de los problemas presentes en el campo de las metaheurísticas es la falta de reutilización \cite{metaheuristics}. Se trata de una falta a nivel de implementación, ya que hay una
tendencia a reimplementar los algoritmos desde 0, impidiendo la reproducibilidad. El progreso científico en cualquier disciplina requiere de la habilidad de construir sobre lo que
otros ya han hecho para poder más lejos. Por ello el algoritmo desarrollado ha sido publicado, además se ha hecho de manera que se pueda usar para cualquier de problema de optimización.
Para ayudar a la reutilización es muy importante que el código sea limpio, mantenible y haya sido testeado, por eso se ha decidido seguir los estándares de arquitectura y código marcados 
por Robert C. Martin \cite{cleanArquitecture2017, cleanCode2008}.

Los lenguajes que se suelen usar para programación genética son: Matlab, C++, Python o Java \cite{languages}. Siendo Python el más utilizado debido a la cantidad de librerías disponibles, 
con algoritmos genéticos listos para usar, o la facilidad para extraer gráficas de los resultados. En general, a la hora de lanzarte a programar un proyecto relacionado
con Aprendizaje Automático o Inteligencia Artificial, es el primer lenguaje que se viene a la cabeza. Sin embargo, en la primera
implementación de este algoritmo se usó Python \cite{merelo_molina_2021} y las prestaciones no fueron muy altas. En el caso de
Julia, no hay tantos paquetes desarrollados, sin embargo su popularidad está aumentando en el mundo de la programación
científica. Una de las principales razones es su unión en 2017 al ``Club Petaflop``, que confirma el buen rendimiento
que puede llegar a ofrecer.  