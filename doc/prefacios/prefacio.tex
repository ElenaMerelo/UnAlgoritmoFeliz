\thispagestyle{empty}

\begin{center}
{\large\bfseries Algoritmos Basados En Población Con Diversidad Mejorada \\ Algoritmo basado en Un Mundo Feliz de Aldous Huxley }\\
\end{center}
\begin{center}
Cecilia Merelo Molina\\
\end{center}

%\vspace{0.7cm}

\vspace{0.5cm}
\noindent{\textbf{Palabras clave}: \textit{software libre}
\vspace{0.7cm}

\noindent{\textbf{Resumen}\\
	

\cleardoublepage

\begin{center}
	{\large\bfseries Algorithm Based In a Diversity-Improved Population}\\
\end{center}
\begin{center}
	Cecilia Merelo Molina\\
\end{center}
\vspace{0.5cm}
\noindent{\textbf{Keywords}: \textit{open source}, \textit{floss}
\vspace{0.7cm}

\noindent{\textbf{Abstract}\\

At the beginning of this year I read A brave new world by Aldous Huxley. The Nobel describes a dystopia, 
which anticipates the development of breeding technology, and how this technology creates the perfect human race. 
Taking into account that then talking about genetic algorithms our goal is to achieve the optimum solution of a problem, 
and this book kind of describes the process for making the “perfect human”, so this similitude is what we will try to develop in this paper. 
The goal is to develop a Genetic algorithm based on the fecundation process of the book and compare it to other algorithms to see how it behaves. 
Investigating how the division in castes affects the diversity in the population.

\cleardoublepage

\thispagestyle{empty}

\noindent\rule[-1ex]{\textwidth}{2pt}\\[4.5ex]

D. \textbf{Tutora/e(s)}, Profesor(a) del ...

\vspace{0.5cm}

\textbf{Informo:}

\vspace{0.5cm}

Que el presente trabajo, titulado \textit{\textbf{Chief}},
ha sido realizado bajo mi supervisión por \textbf{Estudiante}, y autorizo la defensa de dicho trabajo ante el tribunal
que corresponda.

\vspace{0.5cm}

Y para que conste, expiden y firman el presente informe en Granada a Junio de 2018.

\vspace{1cm}

\textbf{El/la director(a)/es: }

\vspace{5cm}

\noindent \textbf{(nombre completo tutor/a/es)}

\chapter*{Agradecimientos}




